Развитие любой науки не происходит плавно: спады и подъёмы мысли, периоды плодотворности и затишья прослеживаются в истории всех дисциплин — от философии до математики. Историческая наука учит нас анализировать события, делать выводы и извлекать из фактов кризисных периодов не только множество позитивных, полезных для практических исследований утверждений, но и отрицательных, предупреждающих. Кризисы в основаниях математики во многом схожи с субъективными кризисами "верований" человека в том смысле, что мы принимаем обоснованные верования как знания. В частности, вопрос релевантности применения математических конструкций в практических целях первоначально ставился в отношении несоизмеримых величин, а позднее и понятий, лежащих в основе математического анализа. Однако, на наш взгляд, эпоха "третьего кризиса" математических оснований, продолжающая тенденции второго кризиса, породила такие фундаментальные вопросы к самой математике и её методам, что ответы на них потребовали глубокой философской рефлексии. Эти философские основания в области как чистой, так и вычислительной математики развивались независимо, но параллельно. В данной работе предлагается рассмотреть кризисный период развития математики в XX веке как разрешившийся феномен и на примере этого материала обобщить ключевые вопросы к знаниям и их основаниям с философской точки зрения, используя формальные средства.
	
	В последние десятилетия считается возможным построить формальную онтологию репрезентации знания таким образом, чтобы она сближала области математических и естественнонаучных рассуждений, а также предоставляла формальный процесс проверки правильности доказательств. Несмотря на отмеченную перспективу, в литературе отсутствует целостное исследование формальной онтологии математических объектов с философской точки зрения, учитывающее не только техническую, но и историческую компоненты развития этой идеи. Существуют работы, предлагающие математические теории в качестве средств репрезентации знания (например, статья А. В. Родина "Истина и доказательство в репрезентации знания" \cite{RodinKovalyov2025}), однако они не описывают такие важные моменты, как структуралистская перспектива этого подхода и его доказательная природа. Аналогично, прикладные исследования (к примеру, "На пути к принципам проектирования онтологий, используемых для обмена знаниями" Томаса Грубера \cite{Gruber1993}) лишь фрагментарно касаются математических теорий, не обеспечивая исторического контекста и не раскрывая философских оснований. В результате создаётся впечатление, что формальные онтологии в математических теориях появляются "из ниоткуда"\  , без преемственности и глубокого осмысления их генезиса.
	
	\textbf{Актуальность} предлагаемой работы определяется тем, что на момент её написания важность и практическая значимость описательного языка формальной онтологии в отношении математических объектов и теорий не была исследована с философской точки зрения в достаточной мере. С одной стороны, сохраняется разрыв между развитием математических структур и методами формализации, а с другой — отсутствует совместное рассмотрение этиологии кризисов оснований математики и развития формальной онтологии. 
	
	В связи с этим ставится \textbf{цель} исследования -- показать, как возникший в XX веке кризис оснований математики привёл к формированию нового структурно-философского взгляда на математические объекты и методы, и далее дать плавное введение в применение аппарата математических теорий, при помощи их представления с перспективы формальной онтологии.
	
	Для достижения выше поставленной цели важным является выполнение следующих \textbf{задач}:
	\begin{enumerate}[leftmargin=1em]
		\item Описать ту историческую перспективу, в которой сложности с реализацией поставленной математической проблемы привели к новому взгляду на математические объекты 
		\item Описать этот новый взгляд на объекты без углубления в математические тонкости
		\item Эксплицировать, как именно структуралистский подход повлиял на исследовательскую перспективу
		\item Обобщить новый взгляд на математические объекты и проблемы, дав его описание на философском языке при помощи построения формальной онтологии
		\item Описать в общих моментах теорию, являющуюся результатами такого взгляда, описать её место в решении первоначальной математической проблемы и возможности ее применения вне математики.
	\end{enumerate}
	
	Таким образом, исследование призвано заполнить существующий пробел в литературе, представив целостную картину развития формальной онтологии математических объектов — от истоков кризисов оснований до современных приложений.
	\textbf{Объектом} исследования является проблема обоснования непротиворечивости математики, сформулированная на рубеже XIX—XX веков и развивавшаяся вплоть до наших дней. \textbf{Предметом} исследования выступает структуралистский взгляд на указанную проблему и его следствия: способы формализации, философские основания и практические приложения формальных онтологий.
	
	Общий объём используемой литературы составляет 52 наименования. В первой главе работы основное внимание уделяется трудам Давида Гильберта, поставившего в 1900 году проблему доказательства непротиворечивости аксиом арифметики, а также предшественникам этой задачи (М. Паш, Д. Пеано, М. Пери, Г. Биркгофф, Г. Гельмгольц, Э. Кристоффель, Л. Кронекер, Г. Фреге, Р. Дедекинд, Г. Кантор). Далее рассматриваются труды Сандерса Маклейна как основу структуралистского взгляда (включая ссылки на Дедекинда, Л. Э. Брауэра, Э. Нётер и У. Лавера), а затем — идеи Курта Гёделя в области оснований математики.
		
	Каждый параграф основной части посвящён самостоятельному аспекту исследуемой темы, основное внимание в первом параграфе первой главы уделяется трудам Давида Гильберта, поставившего на конференции в 1900м году проблему доказательства непротиворечивости аксиом арифметики.  Исследуются истоки этой проблемы, программы её решения и смена перспективы, упоминаются авторы работающие в этой области. Во втором параграфе большая часть внимания уделяется Сандерсу Маклейну, как основоположнику структуралистского взгляда на математику. Упоминаются такие авторы как Р. Дедекинд, Л. Брауэр, Э. Нётер и У. Лавер как те, чьи методы и взгляды на математические объекты лежат в основе структуралистской перспективы. В третьем параграфе первой главы, с описанной ранее структуралистской перспективы исследуется Курт Гёдель, последовательно излагаются его идеи в области оснований математики. Формулировка пути к доказательству непротиворечивости математики даётся в задачных терминах, подхода перенятого из работы \cite{Samohvalov} Самохвалова и Ершова.
	В первом параграфе второй главы сформулированная ранее математическая проблема инициирует Гёделя описать рамку некоторой теории, что рассматривается с философской точки зрение как выстраивание формальной онтологии. Описывается построенная Куртом Гёделем система, которая является элементарной теорией типов. Соответствие, называемое "изоморфизмом Карри-Ховарда" используется для перехода к интерпритации изначальной теории в интуиционистском ключе, с опорой на Гейтинга. Во втором параграфе второй главы описывается минимальное ядро интуиционистской теории типов, теории, которую мы изначально назвали "сильной"\  , необходимой для доказательства непротиворечивости математики. В третьем параграфе второй главы описывается применение сильной теории для автоматизации проверки математических доказательств, но путём анализа этой теории мы обнаруживаем, что она не является пригодной для выполнения изначально поставленной задачи доказательства непротиворечивости всей математики. В четвёртом параграфе второй главы даётся обзор моделей нашей изначальной теории и описывается их применение в не математических целях.
