\documentclass[14pt]{extarticle}
\usepackage[T2A]{fontenc}
\usepackage[utf8]{inputenc}
\usepackage[russian]{babel}

\usepackage{amsmath, amssymb, amsfonts}
\usepackage{geometry}
\usepackage{csquotes}
\usepackage{hyperref}
\usepackage{setspace}
\usepackage{bussproofs}
\usepackage{enumitem}
\usepackage{xcolor} % для выделения цветом

\usepackage{graphicx}
\usepackage{float}            % для [H]
\usepackage{tikz}     
\usepackage{tikz-cd}        % для рисунков
\usepackage{tabularx}
\usepackage{array} % для \arraybackslash
\usetikzlibrary{arrows.meta, positioning, shapes.geometric}
\usepackage{stmaryrd} 
\providecommand\nobreakdash{\hbox{-}\nobreak\hskip0pt}
\allowdisplaybreaks

\usepackage{caption}          % красиво оформляет подписи
\usepackage{booktabs}         % улучшенная верстка таблиц
\sloppy

\renewcommand{\footnotesize}{\fontsize{12pt}{11pt}\selectfont}

\usepackage[
backend=biber,
style=gost-numeric,		%ext-numeric, можно также 'gost-authoryear'
language=russian,
sorting=langnty
]{biblatex}

\addbibresource{bib/kursach.bib} % путь к твоему .bib файлу

% Разрешаем проверять язык записи (langid)
\usepackage{etoolbox}

% Для одиночного постнота (например, \cite[12]{key}):
\DeclareFieldFormat{postnote}{%
	\iffieldequalstr{langid}{russian}
	{c.~#1}% если langid = russian → «с. <номер>»
	{p.~#1}% иначе → «p. <номер>»
}

% Для множественного постнота (например, \cite[12, 34]{key}):
\DeclareFieldFormat{multipostnote}{%
	\iffieldequalstr{langid}{russian}
	{c.~#1}% «с. 12, 34»
	{p.~#1}% «p. 12, 34»
}


% ГОСТ сокращения городов
\DeclareListFormat{location}{%
	\ifstrequal{#1}{Москва}{М.}%
	{\ifstrequal{#1}{Санкт-Петербург}{СПб.}%
		{\ifstrequal{#1}{Ленинград}{Л.}%
			{\ifstrequal{#1}{Минск}{Мн.}%
				{\ifstrequal{#1}{Новосибирск}{Новосиб.}%
					{#1}}}}}%
}

\newcommand{\mycomma}{\addcomma\space}

% --- BOOK ---
\DeclareBibliographyDriver{book}{%
	\printtext{\mkbibemph{\printnames{author}}}%
	\mycomma
	\printfield{title}%
	\mycomma
	\printlist[location]{location}
	\mycomma
	\printfield{year}%
	\finentry
}

% Аналогично для остальных драйверов:

\DeclareBibliographyDriver{article}{%
	\printtext{\mkbibemph{\printnames{author}}}%
	\mycomma
	\printfield{title}%
	\mycomma
	\printfield{journaltitle}%
	\mycomma
	\printlist[location]{location}
	\mycomma
	\printfield{year}%
	\finentry
}


% --- INCOLLECTION ---
\DeclareBibliographyDriver{incollection}{%
	\printtext{\mkbibemph{\printnames{author}}}%
	\mycomma
	\printfield{title}%
	\mycomma
	\printfield{booktitle}%
	\mycomma
	\printlist[location]{location}
	\mycomma
	\printfield{year}%
	\finentry
}

% --- INBOOK ---
\DeclareBibliographyDriver{inbook}{%
	\printtext{\mkbibemph{\printnames{author}}}%
	\mycomma
	\printfield{title}%
	\mycomma
	\printfield{booktitle}%
	\mycomma
	\printlist[location]{location}
	\mycomma
	\printfield{year}%
	\finentry
}

% --- MISC ---
\DeclareBibliographyDriver{misc}{%
	\printtext{\mkbibemph{\printnames{author}}}%
	\mycomma
	\printfield{title}%
	\mycomma
	\printlist[location]{location}
	\mycomma
	\printfield{year}%
	\finentry
}

% --- INPROCEEDINGS ---
\DeclareBibliographyDriver{inproceedings}{%
	\printtext{\mkbibemph{\printnames{author}}}%
	\mycomma
	\printfield{title}%
	\mycomma
	\printfield{booktitle}%
	\mycomma
	\printlist[location]{location}
	\mycomma
	\printfield{year}%
	\finentry
}

\DeclareSortingTemplate{langnty}{
	% 2.1. Сортируем по langid (строку "russian" < "english")
	\sort[direction=descending]{
		\field{langid}
	}
	% 2.2. Внутри каждой языковой группы — по автору (editor, translator, sortname)
	\sort{
		\field{author}
		\field{editor}
		\field{translator}
		\field{sortname}
	}
	% 2.3. Затем — по году (date)
	\sort{
		\field{year}
		\field{date}
	}
	% 2.4. И, наконец, по названию (title)
	\sort{
		\field{title}
	}
}


\geometry{
	a4paper,
	top=2cm,
	bottom=2cm,
	left=3cm,
	right=1.5cm
}

\setlength{\parindent}{12.5mm}
\setlength{\parskip}{0.7em}
\onehalfspacing
\newcommand{\Con}{\mathrm{Con}}


\begin{document}
\begin{titlepage}
	\centering
	\small
	МИНИСТЕРСТВО НАУКИ И ВЫСШЕГО ОБРАЗОВАНИЯ РОССИЙСКОЙ ФЕДЕРАЦИИ
	
	ФЕДЕРАЛЬНОЕ ГОСУДАРСТВЕННОЕ АВТОНОМНОЕ ОБРАЗОВАТЕЛЬНОЕ УЧРЕЖДЕНИЕ ВЫСШЕГО ОБРАЗОВАНИЯ\\
	\textbf{«*Ваша гордость*»}
	
	ОНК «ИНСТИТУТ ОБРАЗОВАНИЯ И ГУМАНИТАРНЫХ НАУК»\\ %к примеру
	ВЫСШАЯ ШКОЛА ФИЛОСОФИИ, ИСТОРИИ И СОЦИАЛЬНЫХ НАУК
	
	\vfill
	
	\large
	Мамин симпотяга\\[0.5em]
	\textbf{Почему коты так красивы}\\[0.5em]
	
	\small
	Направление \\
	Профиль «...»\\
	Курсовая работа
	
	\vfill
	\begin{flushright}
	Научный руководитель:\\
	...\\ %регалии
	\textbf{Лаврентий Михайлович Берия}\\[2em]
	\end{flushright}
	
	\begin{flushleft}
	Показатель уникальности \\
	текста составляет \underline{\hspace{2cm}} \%\\[1em]
	«\underline{\hspace{1cm}}» \underline{\hspace{3cm}} 2025 г.\\
	\underline{\hspace{6cm}}\\
	\textit{подпись должностного лица, ответственного \\
		 за проверку работы в системе «Антиплагиат. Вуз»}
	
	\vspace{2em}
	
	Работа допущена к защите\\
	«\underline{\hspace{1cm}}» \underline{\hspace{3cm}} 2025 г.\\
	\underline{\hspace{6cm}}\\
	\textit{И. О. Директора \\
	чего-то там}\\
	\textbf{тко-то там}	
	\vfill
	\end{flushleft}
	Город \\
	год
\end{titlepage}


\newpage
\tableofcontents
\thispagestyle{empty} 
\newpage

	\section*{Введение}
	\addcontentsline{toc}{section}{Введение}
	
	\input{../images/cat.png}

	
	\newpage
	\section*{Глава 1. Основания математики}
	\addcontentsline{toc}{section}{Глава 1. Основания математики}
	
	\subsection*{1.1 Описание ситуации третьего кризиса}
	\addcontentsline{toc}{subsection}{1.1 Описание ситуации третьего кризиса}
	
	В пятом параграфе первой главы "Оснований теории множеств" Френкеля и Бар-Хиллеля \cite[26]{SetTheory} кратко представляются три ситуации из истории философии математики, сходные в том, что появление качественно новых методов порождало вопросы к их основаниям, требующим не столько математических, сколько философских ответов и построений. Эта апелляция к истории должна была предоставить более позитивный взгляд на современное (на тот момент) положение философии математики. Необходимость такого "подбадривания" связана с получением множества ограничивающих результатов для формальных систем.
	
	Всё началось с вопросов к постулатом Евклидовой геометрии, встала цель построения геометрической теории с чистого листа \cite[288]{rodin2020}. Это значит, что в этом случае очевидные факты нашей (пространственной) интуиции не могут просто подразумеваться как в прикладной геометрии, а должны быть явно описаны и аксиоматизированы. Пионерскими работами в этой области можно считать \textit{"Лекции по новейшей геометрии" (Vorlesungen über neuere Geometrie)} Морица Паша, опубликованную в 1882 и развитие этой работы Джузеппе Пеано \textit{"Лекции по новой геометрии"}\footnote {Имеется английский перевод \cite{JeanVanHeijenoort}} 1889-го года. Пеано развил начала аксиоматического метода Паша, который был не вполне последовательным. Так, для аксиоматизации геометрии он воплотил нетривиальные (на тот момент) интуиции по поводу постулирования лишь примитивных символов (Kernbegriffe), по его мнению единственное место, которое мы можем оставить для интуиции в формализованной теории, это интерпретация этих символов и их отношений в примитивных аксиомах. Также стоит отметить, что выбор примитивных символов обосновывался тем, что мы можем иметь вполне хорошие интуиции (внутренние созерцания) относительно таких объектов как точка, но не относительно бесконечно продолжающейся прямой, потому аксиомы о прямых были заменены аксиомами об отрезках. Пеано же усмотрел проблему в выборе примитивных объектов и разорвал связь с эмпирикой, выстроив полностью формальную систему, записанную в собственной символической нотации на основании понятия точки, как примитивного объекта и отношения посредничества (betweenness).
	
	Позже к работам в основаниях геометрии подключилось множество передовых исследователей, более того, работы в этой области публикуются до сих пор. Мы отметим несколько важных работ и школ, к примеру Марио Пери и Итальянская школа геометров, в которую входили также Цезаре Бурали-Форти, Александр Падоа и Джино Фано, стояли в основаниях топологической аксиоматизации геометрии. Используя нотацию Пеано они избавились от отношения посредничества, заменив его понятием движения. Более поздние аксиоматические построения Биркгофффа \cite{Birkhoff1932} имели лишь 4 постулата, основывались на системе натуральных чисел и служили целью дать такие основания, в которых все постулаты могут быть экспериментально верифицированы, путём построения с помощью линейки и транспортира. Аксиоматика Саундерса МакЛейна \cite{MacLane1959} является развитием идей Биркгоффа, его привнесением в теорию является введение функции, ассоциирующей натуральные числа с сегментами линии. 
	
	Особое место занимает работа Давида Гильберта \textit{"Основания Геометрии" }(1899), в которой он не только предложил строгое разделение между примитивными понятиями и аксиомами, но и настаивал на их полной формальной абстракции: такие термины, как точка или прямая, в этой системе не обязаны иметь привычное геометрическое значение — они служат пустыми символами, содержание которым придаётся только в рамках модели, удовлетворяющей системе аксиом. Гильберт выделяет несколько групп аксиом: инцидентности, порядка, конгруэнтности, непрерывности и, начиная с французского издания, аксиому полноты, запрещающую расширение геометрической структуры без нарушения остальных аксиом. Подход Гильберта также позволяет дать развёрнутое описание понятиям истинности и существования, в письме Фреге он пишет\footnote{Перевод по указанному изданию А.В. Родина можно найти в \cite[290]{rodin2020}.} \textit{“Как только я установил аксиому, она существует и истинна. Если произвольный набор аксиом и их всевозможных следствий не порождает противоречия, значит, они все истинны и вещи, определяемые этими аксиомами, существуют. На наш взгляд, это и есть критерий истинности и существования.”} \cite[39]{frege1980correspondence} Аксиомы Гильберта, сказываемые о "мыслевещах" (Gedankendinge) в этом плане истинны, как способность непротиворечиво помыслить объекты и их отношения конкретно, без приписывания им дополнительных свойств. Способность к таков рода представлениям можно понимать по аналогии с тем, что Кант называет интуицией в "доктрине о методе"(\textit{Математика не может ничего достичь с помощью одного лишь понятия, но немедленно обращается к интуиции, в которой она рассматривает понятие в конкретном виде, хотя и не эмпирически; скорее, математика рассматривает понятие только в интуиции, которую она представляет априори — то есть в интуиции, которую она сконструировала, и в которой всё, что следует из универсальных условий конструкции, также должно быть универсально применимо к объекту сконструированного понятия}) \cite[395]{kant1994critique}. Позже эти неясные идеи будут преобразованы Альфредом Тарским в формальную теорию моделей \cite{Tarski1956}.
	
	Теория Гильберта не была строго формальной, язык на котором она была сформирована не подразумевал необходимой математической строгости, что соответственно позволило сделать следующий шаг Альфреду Тарскому. Он был был вдохновлён работами Марио Пьери, символической логикой Пеано, а также опирался на Эрлагенскую программу Клейна (отождествление логических понятий с инвариантами всех перестановок элементов некоторого множества) и в работе 1926 года предложил систему аксиом для элементарной евклидовой геометрии, на основании первопорядковой логики. В ней используются только точки как примитивные объекты и два примитивных отношения — "между" и "конгруэнтность". В отличие от систем Гильберта и Биркгоффа, в которых линии и углы фигурируют как отдельные элементы или определяются на основе других объектов, в системе Тарского все объекты сводятся к точкам, а все высказывания — к формулам логики первого порядка. Этот подход позволил системе Тарского продемонстрировать исключительные метатеоретические свойства. Помимо того, что она является полной, непротиворечивой и разрешимой (благодаря тому, что в отличие от теорий арифметики, не обладает достаточной выразительной силой для кодирования всей арифметики Робинсона и тем самым не попадает под действие теорем Гёделя о неполноте), в ней также выразима теория гиперболической геометрии (строящаяся путём отрицания постулата параллельности Евклида), а также абсолютная геометрия (если полностью элиминировать из теории постулат параллельности).
	И пусть система Тарского неудобна для практического доказательства геометрических фактов, она всё же послужила началом сближения философии и формальных методов. Теория моделей фактически служит такой абстракцией, в которой выражение оснований (математических интуиций) необходимо постулируется и используется.
	
	Но несмотря на успех в основаниях геометрии, с обоснованием логики и арифметики возникли проблемы. Сам Гильберт пишет \textit{"Трудности, встречающиеся при обосновании арифметики, частично действительно отличаются от тех, которые надо было преодолеть при обосновании геометрии. При исследовании основ геометрии можно было обойти некоторые затруднения чисто арифметической природы; но при обосновании арифметики ссылка на другую основную дисциплину становится уже недопустимой"} \cite[399]{Gilbert_izbrannoe}. 
	Из выступления Гильберта на третьем математическом конгрессе в 1905 году \cite{Gilbert1924} известно, что он подверг критике ряд подходов к обоснованию числовых понятий. Л. Кронекер был назван догматиком за неспособность осознать необходимость обоснования самого понятия целого числа. Г. Гельмгольц, представленный как эмпирик, критиковался за попытку вывести понятие бесконечного числа исключительно из опыта, что, по мнению Гильберта, невозможно, поскольку опыт всегда ограничен конечным числом объектов. Э. Б. Кристоффель и противники Кронекера, пытавшиеся обосновать существование иррациональных чисел через "положительные" свойства, были названы оппортунистами, а их подход — бесплодным. Хотя Гильберт признавал, что Г. Фреге в \textit{Основоположения арифметики} (1884) глубже понимал суть проблемы, он всё же отвергал его метод из-за неразрешённого парадокса Рассела и неспособности логики удовлетворить требованиям теории множеств. Метод Р. Дедекинда, названный трансцендентальным, также отвергался как ненадёжный из-за своей философской природы и апелляции к совокупности всех вещей, что, по мнению Гильберта, приводит к противоречиям. Выше всех оценивались заслуги Г. Кантора и его \textit{К обоснованию учения о трансфинитных множествах}\footnote{Кантор Г. Труды по теории множеств / Перевод Ф. А. Медведева и П. С. Юшкевича. — М.: Наука, 1985. — С. 173.} (1895), однако и здесь Гильберт указывал на недостаточную строгость в разграничении консистентных  (синтаксически непротиворечивых) и неконсистентных множеств, что не даёт объективной уверенности.
	
	Гильберт, осмысляя успех в основаниях геометрии, выделяет оппозицию двух методов: генетического и аксиоматического. В качестве примеров использования генетического метода он указывает на введение понятия действительного числа с помощью последовательностей Коши и сечений Дедекинда, а также на возможность определения рациональных и целых чисел как классов эквивалентности пар натуральных чисел. Во всех этих случаях новое математическое понятие порождается (erzeugt) из набора уже известных. Из работы Гильберта 1924 года мы можем подметить более полный фрагмент разъясняющий  его аксиоматический метод \textit{"В то время как при конструктивном способе построения объекты рассматриваемой теории вводятся только как вещи определенного вида, в аксиоматической теории нам приходится иметь дело с некоторой фиксированной системой вещей (или даже с несколькими такими системами), вводимой в качестве области субъектов для всех тех предикатов, из которых строятся высказывания этой теории"} \cite[24]{GilbertBernays}. Генетический метод "конструирует" свои объекты из элементарных, путём рекурсивного применения набора данных операций, в то время как формальная аксиоматическая теория включает в себя понятие интерпретации (модели)\footnote{Под моделью мы понимаем носитель интерпретации некоторого замкнутого множества истинных формул в языке, сигнатура которого включает символ равенства, удовлетворяющий аксиомам рефлексивности, симметричности и транзитивности, как и каждый функциональный символ, при замене аргументов на равные, см. \cite[174]{VereshchaginShen}.}, которое в свою очередь требует допущения о том, что используемая модель в каком-то подходящем смысле слова существует. В этом смысле аксиомы обладают некоторой "экзистенциальной формой"\  , которую могут имплементировать подходящие объекты, описываемые данной моделью. Таким образом единственным дополнительным, помимо формально теоретических, допущением в аксиоматических теориях является допущение существования модели. Важной проблемой в этой связи авторы усматривают случай существования бесконечных моделей, ведь существование модели теории влечёт её непротиворечивость. Соответственно этому формулируется основное направление  так называемой программы Гильберта — построить математическую теорию, которая позволяла бы доказать или опровергнуть утверждение о непротиворечивости данной формальной аксиоматической теории с помощью финитарных конструктивных средств. Заметим, что аксиоматическая теория не исключает содержательных генетических методов построения, но из такого рода методов, допускает только финитарные (конечные символьные). Проблему же доказательства непротиворечивости аксиоматической теории, Гильберт относит к области метаматематики, которую, во избежании порочного круга в основаниях, предлагает развивать при помощи традиционного генетического метода. 
	
	Однако уже в 1931 году стало ясно, что Гильбертовская программа требует пересмотра. Это следует из результатов Курта Гёделя, опубликованных в работе \textit{"О принципиально неразрешимых положениях в системе Principia Mathematica и родственных ей системах"}\footnote{Сборник К. Гёдель Избранные труды, издательство "Наука"\  , Москва, 1989., А. И. Малцев.} и дополняется работой Альфреда Тарского "\textit{Понятие истины в языках дедуктивных наук}"  1936 года\footnote{Сборник Философия и логика Львовско-Варшавской школы, издательство РОСПЭН, Москва, 1999., перевод с польского В. Л. Васюков.}.
	
	
	\vspace{0.5em}
	Первый результат Гёделя был упомянут ещё в виде реплики, на научном семинаре в г. Кенигсберге, организованном редакцией журнала "Erkenntnis" \cite{pushkarsky2009}. Разъяснение этих реплик было напечатано в "Добавлении" к разделу "Дискуссии об основаниях математики"\ , но к тому моменту результаты Гёделя уже были известны из изложения на заседании Венской академии наук. 
	Первая теорема утверждает, что для любой рекурсивно перечислимой, непротиворечивой, достаточно выразительной формальной системы $T$, включающей арифметику натуральных чисел (например, арифметику Пеано), существует формула $\varphi$, такая что $T \nvdash \varphi$ и $T \nvdash \neg\varphi$, при этом $\varphi$ интерпретируется как истинное арифметическое высказывание. То есть, система $T$ неполна: она не может доказать все истинные утверждения арифметики. Суть доказательства основывается на \textit{диагональной лемме}, позволяющей построить формулу, утверждающую о собственной недоказуемости внутри системы. Вторая теорема была представлена вслед за первой, опубликованная в 1932 году, она усиливает предыдущее утверждение, показывая, что если любая, достаточно богатая математическая система $T$, включающая арифметику целых чисел, действительно непротиворечива, то это невозможно доказать средствами самой системы, то есть $T \nvdash \operatorname{Con}(T)$\footnote{Здесь и далее мы используем Con как обозначение непротиворечивости.}. Это означает, что в рамках самой формальной арифметики невозможно финитарными средствами установить её непротиворечивость, как того требовала гильбертовская программа. Метаматематическое доказательство непротиворечивости должно использовать более мощную систему, выходящую за пределы рассматриваемой.Тарский воспользовался мощным доказательным аппаратом, предложенным Геделем и дополнил его результаты. Его работа показывает фундаментальное ограничение любых формализованных языков, включающих арифметику -- множество всех истинных арифметических утверждений не является арифметически определимым, т.е. не существует формулы в языке арифметики, которая бы точно кодировала предикат истинности. Формально, если $L$ — формальный язык, в котором можно выразить достаточную часть арифметики, то не существует предиката $\mathrm{True}(x)$, такого что $\mathrm{True}(\ulcorner \varphi \urcorner) \leftrightarrow \varphi$ для всех формул $\varphi$ в $L$, где $\ulcorner \varphi \urcorner$ — гёделев код формулы. Это ограничение связано, в частности, с невозможностью разрешения самореференции внутри системы и логически обосновывает необходимость метаязыка для выражения истины.
	
	Гильберт провозгласил высокую цель аксиоматизации всей математики, но видел её достижение в доказательстве полноты и непротиворечивости арифметики натуральных чисел. Результаты, показывающие что существование модели и финитарная доказуемость непротиворечивости не могут быть гарантированы средствами одной и той же формальной системы, являлись не фатальными, но требовали новой философской интерпретации формальной онтологии математических объектов.
	
	\subsection*{1.2 С точки зрения математического структурализма}
	\addcontentsline{toc}{subsection}{1.2 С точки зрения математического структурализма}
	
	В.А. Шапошников в статье "миф о трёх кризисах в основаниях математики" делегитимизирует кризисную часть в описании всех трёх исторических отрезков. Он верно подмечает, что логические и теоретико-множественные парадоксы не воспринимались, к примеру, тем же Гёделем, как сильно ограничивающие, по его мнению они представляют серьёзную проблему только в областях логики и эпистемологии, т. е. Философии \cite[180]{Godel1947} В то же время работы самого Гёделя скорее даже положительно повлияли на плодотворность работающих математиков. По мнению Шапошникова \cite{shaposhnikov2021}, основной конфликт, обозначенный Германом Вейлем в 1921 году имел в своей основе противостояние двух конкретных программ — концептуальный формализм Гильберта и конструктивный\footnote{О конструктивизме Брауэра см. \cite{BrouwerPosy}.} интуиционизм Брауэра. 
	Но если Гильберт стремился к полному аксиоматическому обоснованию всех разделов математики, а Брауэр, напротив, отстаивал интуиционизм и отвергал классическую логику, то структуралистский подход предложил иное основание: не в объектах как таковых, а в отношениях между ними, в структурах, в которые они включены.
	
	Симптоматично, что именно в это время происходит важнейший поворот в понимании натуральных чисел, хотя ещё в 1888 году Рихард Дедекинд в своей работе \textit{"Что такое числа и для чего они служат"} \cite{Dedekind2015} предложил аксиоматическое определение натурального ряда через вложенное отображение и начальный элемент. Ключевой стала его теорема 132, доказывающая, что всякий „просто“ бесконечный ряд, содержащий один элемент и замкнутый относительно определённого отображения, изоморфен ряду натуральных чисел. Это была одна из первых внятных формулировок идеи, что не носители, а структура определяет математический объект — число как место в структуре, а не как нечто самодовлеющее. С исторической точки зрения ни подход Брауэра, ни противолежащие ему идеи Гильберта не были в полном смысле независимо реализованы в программе унивалентных\footnote{Предложенное Владимиром Воеводским название программы по построению новых оснований математики, преследующей также прагматическую цель компьютерной автоматизации проверки математических доказательств.} оснований, но стоит отметить, что именно в Геттингене в то время стали развиваться практические приложения структуралистского подхода.  
	
	Говоря о математическом структурализме следует оговорится, что не существует такой вещи как один канонический философский подход, имплементируемый разными исследователями. Именно поэтому мы не можем заострить внимание только на взглядах одного из теоретиков, проследив их развитие и влияние на формирование онтологии математических объектов, оказавшейся наиболее плодовитой, нет, исторически это был очень комплексный процесс. Но для понимания всё же можно поддаться категоризации, выдвинув общий структуралистский тезис, который звучит как: "\textit{изоморфные объекты можно рассматривать как равные}" \cite{Awodey2014}. В этом выдвинутом С. Аводеем принципе прослеживаются методологические инсайты Эмми Нётер, чей принцип Вольфганг Крулль передаёт как стремление "\textit{рассматривать алгебраические структуры настолько, насколько это возможно, с точки зрения их изоморфизмов}" \cite[4]{Lambek_1970}. Согласно этому принципу, важна не природа самих элементов, а отношения и операции между ними, которые сохраняются при изоморфизмах. Тем самым структура становится первичной, а носители — вторичными. Нётер сделала возможным переход от традиционного "контентного" взгляда на математические объекты к "отношенческому"\ , основанному на морфизмах (см. \cite{Noether}). Именно в таком ключе были сформулированы первые идеи того, что позже стало частью алгоритмической логики, структурного построения доказательств и логических исчислений. Эти идеи повлияли и на Саундерса Маклейна (1909-2005), идеи которого для формализации математики предполагают редукционистский подход к математическим объектам, и в дополненном виде служат основой для автоматической проверки математических доказательств. 
	
	Он посещал еженедельные лекции Дэвида Гильберта по философии в Геттингене в 1931 году.  Снимая комнату в доме Германа Вейля, он работал с ним над его книгой \textit{"Философия математики и естествознания"} (1927). Формирование его взглядов происходило схожим с Куйаном образом, решающее влияние на них, как на студентов, оказала \textit{Principia Mathematica}, во время обучения в престижных вузах (Маклейн обучался в Йеле, Куайн в Оберлине соответственно). Оба они были членами-основателями ассоциации символической логики и преподавали в Гарварде по 1947 год, пусть и на разных факультетах. При этом всём, Маклейн считал, что "внушительный вес принципии продолжал искажать взгляды Куайна на философию математики" \cite[152]{maclane1997}; и он отвергал "излишнюю озабоченность Куайна логикой как таковой" \cite[443]{maclane1986}
	
	Маклейн, ещё до окончательной формализации своих взглядов на теорию категорий, являющуюся важной составляющей математического структурализма, видел будущее математики именно в подходе группы Бурбаки, абстрагирующей числа до роли элемента в структуре (в их случае моноида). При этом, на момент создания группы, у Маклейна уже имелась публикация \textit{"Логический анализ математических структур"} (1935), а многие из участников группы Бурбаки (к примеру Пьер Картье, Лоран Шварц, Александр Гротендик, Анри Картан, Самуэль Эйленберг) интересовались теорией категорий. 
	
	Высоко оценена Саундерсом Маклейном была также категорная интерпретация арифметики в работах Уильяма Лавера. Он разработал структурную теорию множеств ETCS (Elementary Theory of the Category of Sets), в которой множество не определяется как определённый набор элементов, а как объект в категории с определёнными структурными свойствами. Именно у Лавера (см. \cite{Lawvere1964}) мы находим новый подход к формализации, в котором многие парадоксы, связанные с теорией множеств, логикой и метафизикой числа (вопросы исследуемые Бенацерафом), перестают быть проблемами. 
	
	Бенацераф, напомним, cформулировал (см. \cite{benacerraf1965}) известную проблему редукций множественных представлений математических объектов: почему натуральное число 3, заданное как \texttt{{$\{\{\emptyset\}\}$}} (нумералы\footnote{Калька с английского numeral, что в свою очередь является переводом изначального обозначения введённого Г.Кантором (Anzahl). Позже, Кантор стал обозначать порядковый тип конечного вполне упорядоченного множества термином Ordnungszahl, что и закрепилось в русском переводе как "ординал".} Цермелло), эквивалентно числу 3, заданному как \texttt{{$\{\emptyset, \{\emptyset\}, \{\emptyset, \{\emptyset\}\}\}$}} (конечные ординалы Фон Неймана), если многие, выходящие за рамки обычной арифметики, свойства этих представлений чисел не тождественны (к примеру количество подмножеств данного множества)? Лавер, через категориальную теорию и особенно через интерпретацию логики в терминах топосов, показывает, что такие вопросы — следствие чрезмерной привязки к "вещности" объектов и игнорирование структурного контекста, в котором они работают.

	Яркой иллюстрацией выводов Лавера (взятая из \cite{reck2020}) служит рассмотрение автоморфизмов комплексных сопряжений. При рассмотрении сопряжения с алгебраическим контекстом, мы можем сказать, что оно является автоморфизмом поля комплексных чисел. В таком случае мы не проводим различия между $i$ и $-i$. В то же время в контексте комплексного анализа мы понимаем изоморфизм поля комплексных чисел на себя отличным от $\text{id}_{1S}: S \to S$. В таком случае мы используем голоморфное\footnote{Т.е. отображение заданное функцией от комплексного переменного, определённое на открытом подмножестве комплексной плоскости.} отображение, геометрически различающее $i$ и $-i$.
	
	На этом фоне возникает вопрос о роли структуры в практической математике. А. Л. Городенцев, к примеру, как практикующий и преподающий математик, противопоставляет аксиоматический подход — построение понятий точки, прямой и т.п. с помощью аксиом — \textit{аналитическому} подходу. Последний, по его определению, даёт всем используемым объектам явные определения, основанные на понятии числа. При этом даже такие фундаментальные конструкции, как аффинная плоскость, зависят от понятия поля, которое, в свою очередь, задаётся аксиоматически — опираясь на структуру натуральных чисел и операций над ними. Но даже для практики преподавания теория категорий не лишена своей полезности, ведь по расхожему выражению, позволяет взглянуть на математику с высоты птичьего полёта, т. е. подметить те особенности и паттерны из разных областей, которые не заметны при обыденной работе.
	
	Также отметим и некоторый философский аспект теории категорий. Во время работы Маклейна в университете в среде математиков увлечённых вопросами оснований был в почёте Гуссерль (что не удивительно, учитывая что тот обучался у Кроненкера и Вейерштрасса, а позже и у Брентано, защитив предварительно докторскую по теории вариационного исчисления). Соответственно "феноменологический" язык, выделяемый Карнапом как один из способов описания значений слов, был перенят Маклейном также, как и термин функтор. При этом Гуссерль некоторое время пребывал в Гёттингене, после чего его место занял Мориц Гейгер, избавивший теорию Гуссерля от идеализма, но продолжавший развивать феноменологический метод  \cite[200]{spiegelberg2012phenomenological}. Это развитие вылилось в работу 1924го года, в которой явно выражен структурный подход, осмысляющий и дорабатывающий идеи Гильберта, путём связывания аксиом и „идей“ феноменологическим образом. При этом не был забыт и алгебраический метод  Эмми Нётер, описывающийся ей самой как „прогрессирующие исчисления“. Эту ветвь развития оснований продвигал Герман Ганкель. 
	Cтруктурализм позволил освободиться от некоторых псевдопроблем, порождённых чрезмерным реализмом в отношении математических объектов, и сосредоточиться на действительно продуктивных вопросах — о природе изоморфизмов, о категориях и морфизмах, о построении теорий, способных внятно говорить о самих себе. А все эти вопросы, уже в то время, рассматривались с оглядкой на алгебраический метод Эмми Нётер.
	
	\subsection*{1.3. Место для структурализма в решении задачи непротиворечивости математики}
	\addcontentsline{toc}{subsection}{1.3. Место для структурализма в решении задачи непротиворечивости математики}
	
	Курта Гёделя занимала идея доказательства непротиворечивости математики ещё долгое время после публикации известных ограничивающих результатов, также особенно прослеживается его согласие с Гильбертом в том, что доказательство должно быть дано финитарными средствами. В своих поздних работах, о которых мы поговорим позднее, он уточняет эту перспективу.
	Для того что бы понять как Гёдель ставил задачу, обратимся к тексту, состоящему из записей лекции, прочитанной в Вене 29 января 1938 года на семинаре, организованном Эдгаром Зильселем \cite{Godel1938a}. Обсуждение проблемы доказательства непротиворечивости математических систем начинается с выдвижения таких тезисов:
	\smallskip
	\begin{enumerate}
		\item Доказательство непротиворечивости системы  возможно только в случае её сведения к более слабой, а именно 
		\[
		\Con S \implies \Con T \text{ доказуемой в } S \quad (\text{ где } S \text{ это подсистема } T)
		\]
		\[
		\Con T \text{ доказуема в } S
		\]
		\item Программа обоснования математики не может быть проведена с установкой о природе математического знания как беспредпосылочного знания
		\item Вопрос о доказательстве непротиворечивости математики может быть представлен как вопрос представления такой системы \( S \), посредством которой система \( T \) оказывается непротиворечивой
		\item Эпистемологическая сторона вопроса предполагает, что доказательство рассматривается как приемлемое только если оно сводится к уже обоснованной части, либо к тому, что хоть и не является такой частью, всё же является более очевидным и надёжным в смысле нашей субъективной уверенности\footnote{Понятие субъективной уверенности довольно проблематическое, поэтому Гёдель принимает предпосылку о том, что конструктивные системы в целом более удовлетворительны для доказательства.}. 
	\end{enumerate}
	
	Поясним первый пункт в задачных терминах, по аналогии\footnote{Клементий Фёдорович и Юрий Леонидович в своей работе показали плодотворность взгляда на теории с точки зрения вопроса решения задач, для которых они предназначаются, но особая терминология, введённая ими в работе, излишня для нас на данном этапе.} с тем, как это было сделано в совместной работе Самохвалова и Ершова \cite{Samohvalov}.
	Определим стандартный арифметический предикат доказуемости в \(S\) как \(\text{Pr}_s\), а запись	\(\ulcorner \phi \urcorner \text{ будет означать Гёделев номер формулы } \phi\). Тогда пусть \(S\) — некоторая "слабая" теория первого порядка, не содержащая в своём языке полного рекурсивного фрагмента арифметики и, не удовлетворяющая хотя бы одному из трёх условий Гильберта–Бернайса (см. \cite[27]{Samohvalov}). Обозначим через
	\[
	\Phi_1(S) = \left\{ \varphi(x) \;\middle|\;
	\begin{aligned}
		& \text{где } \varphi \text{ — формула языка } S, \\
		&\text{имеющая ровно одну свободную переменную } x
	\end{aligned}
	\right\}
	\]

	
	\smallskip
	
	\noindent Рассмотрим далее множество пар "теория–задача"
	\[
	D_S = \{ (T, \varphi(x)) \mid T \supseteq S \text{ — расширение } S, \; \varphi(x) \in \Phi_1(S) \}
	\] и положим, что множество этих пар не пусто \(D_S\neq\varnothing\).
	
	\smallskip
	
	\noindent Для каждой пары \((T,\varphi)\in D_S\) определим предикат "\(T\) непротиворечива в \(S\) относительно задачи \(\varphi\)":
	\[
	\Con_S\bigl(T,\varphi\bigr)
	\;\longleftrightarrow\;
	\exists\,U\;\bigl(U\succ T\;\wedge\;
	U\vdash_S\;\neg\,\exists y\;\text{Pr}_s\bigl(\ulcorner0=1\urcorner\bigr)\bigr),
	\]
	где \(U\succ T\) означает, что \(U\) — более сильная теория, чем \(T\); \(\vdash_S\) — отношение выводимости, формализованное в \(S\) через стандартный предикат \(\text{Pr}_s(z)\), а через \(\ulcorner0=1\urcorner\) обозначен Гёделев номер противоречия.
	
	\smallskip
	
	\noindent Основное условие "сводимости" задачи по доказательству непротиворечивости состоит в том, что бы предъявить существование единой, "более сильной" теории \(U\), в которой для всех пар теория-задача \((T,\varphi)\in D_S\) доказуема непротиворечивость изначальной системы средствами её более слабой подсистемы
	\(
	U\vdash_S\;\Con_S\bigl(T,\varphi\bigr).
	\)	Таким образом мы свели рассматриваемый вопрос к предъявлению такой сильной теории, в которой возможно доказательство непротиворечивости фрагмента математики средствами более слабой теории. При этом нам теперь нужно определить вид этой более сильной теории. В первую очередь она должна быть конструктивна, для того то бы доказательства в ней были удовлетворительны (см. сноску 13). 
	
	Далее, Гедель задаёт некоторую рамку теории \cite[91]{Godel1938}, в которой изначально определяет, что \textit{примитивные операции и отношения этой теории должны быть разрешимы и вычислимы.} И начиная с этого пункта мы можем говорить о влиянии идей математического структурализма на программу обоснования непротиворечивости математических теорий. Это было поистине важным и небанальным нововведением, особенно если сравнить его с точкой зрения Тьюринга, чьё имя сейчас уже стало нарицательным в теории вычислимости. В своей работе выпущенной на 2 года раньше \cite{turing1936} А.Тьюринг вводит понятие машины и определяет вычислимость главным образом как абстрактный критерий механического расчёта, не привязываясь к конкретному синтаксису какой-либо формальной теории, он демонстрирует "вычислимость" как универсальную абстракцию, Гёдель же показывает, как это свойство можно инкорпорировать в структуру сильной теории. И конечно же в этой идее легко прослеживается линия Эмми Нётер, выставление акцента на алгебраические свойства системы, формулировка правил в терминах сохранения инвариантов (как мы увидим позже) и буквальная имплементация метода прогрессирующих исчислений, но уже не для исследования заданной структуры, а для её формирования.	
	
	Также Гёдель заметил, что поддержание "финитарности" средств теории возможно, если в теории применима полная индукция, а определения новых понятий в теории задаются рекурсивно, соответственно пункт про вычислимость и разрешимость примитивных операций и отношений был необходим ему также и для этого. Второй пункт уже чисто технический, в нём он запрещает применение экзистенциальных кванторов к лямбда абстракции над выражением, равно как и к примитивным объектам теории, при этом лямбда выражения также не могут появляться  пропозициональных исчислениях.  То есть фактически мы определяем правило, согласно которому ограничиваем область квантификации только выражениями со свободными переменными (переменная соответственно может быть связной также в том смысле, что она объемлема вышележащей операцией абстракции), ведь при квантификации по функциям мы сталкиваемся с онтологическим вопросом предъявления существования правила, что портит математическую строгость рассуждений. 
	\newpage
	\section*{Глава 2. Формальная онтология}
	\addcontentsline{toc}{section}{Глава 2. Формальная онтология}
	\subsection*{2.1. Дизайн\footnote{Мы используем слово "дизайн" в соответствии с устоявшейся традицией уоптребления этого термина с 90-х годов (cм. пункт «Ontology Design» в разделе “Defining an Ontology” \cite[5–7]{gruber1993translation}.} формальной онтологии}
	\addcontentsline{toc}{subsection}{2.1. Дизайн формальной онтологии}
	
	Определим формальную онтологию согласно Н.Кокчиарелли \cite{Cocchiarella1996}, как некоторую логико-грамматическую структуру вместе со способами дедуктивного преобразования внутри этой структуры. Мы используем понятие формальной онтологии в первую очередь из-за того, что логическая теория, представленная как грамматика, определяет только структурные отношения и позволяет осмысленно описывать многие классы объектов. В таком случае дизайном формальной онтологии назовём множество критериев, согласно которым мы устанавливаем необходимую описательную рамку.
	
	Кокчиарелли утверждает, что для дизайна формальной онтологии является существенным то, как мы специфицируем предикацию в изначальной логической теории \cite[28]{Cocchiarella1996}. Ранние теории специфицировали связку предикации посредством интерпретации, соответственно заранее ограничивая описательные возможности теории, мы же следуя рассуждениям Гёделя, специфицируем предикацию только по построению, понимая её как примитивную операцию, которая должна быть разрешима. С точки зрения построения формальной онтологии такое задание абсолютно валидно, поэтому будем отождествлять описательную рамку конструктивной теории с критериями построения формальной онтологии.
	
	Дополним рассуждения Гёделя. В работе \cite{Gruber1993} есть интересное замечание, находящееся в первой сноске, что в традиционном логическом смысле, онтологии отождествляются только с таксономическими иерархиями классов, а определения примитивных терминов и правил не ограничивают их возможные интерпретации. Как мы видим, такой критике нельзя подвергнуть Геделя, в современных терминах можно сказать, что он специфицирует выстраиваемую им онтологию, и более того, можно говорить о том что именно структуралистский подход к проблеме обеспечивает чёткую спецификацию онтологии за счёт внимания к её синтаксическим примитивам, которые в совокупности ограничивают класс допустимых интерпретаций, гарантируя тем самым однозначность и проверяемость. Для того что бы эти критерии были полны, к ним следует также добавить 1) потенциальную расширяемость языка теории\footnote{Она должна предлагать концептуальную основу для ряда предполагаемых задач, а представление должно быть построено таким образом, чтобы можно было монотонно расширять и специализировать онтологию \cite[3]{Gruber1993}.}, далее обозначаемую как \textbf{ext} и 2) минимальные онтологические обязательства языка этой теории, далее обозначаемые как \textbf{moc}.
	
	Уточним критерий moc. Лексикой мы называем понятия, постулируемые теорией, необходимые для решения проблемы. Критерий минимальности онтологических обязательств требует, чтобы в формальную систему включались лишь те понятия и объекты, которые невоз­можно устранить без утраты возможности решить основную задачу. Соответственно в сильных теориях, служащих цели доказательства непротиворечивости могут постулироваться разные лингвистические сущности. К примеру, в той же работе Самохвалова и Ершова, постулируется \cite[38]{Samohvalov} существование таких элементарных объектов как 
	
	\[
	E = \{ L, \mathbf{\Phi}, U \}
	\]
	
	где \( U \) — \( \Phi \)-полезная теория, если и только если для некоторого \( i \in I \) (где \( I \) — элемент \( S_i \) семейства \({\Sigma}(\mathbf{\Phi}) \)) выполняется условие: \( \exists i \in I ( UA_{\mathbf{\Phi}} \subseteq S_i )\), т.е. априорная компонента теории \( U \) объемлет \( S_i \).
	Как видно, язык такой теории не удовлетворяет второму выдвинутому критерию, ведь он оказывается гиперспецифицированным. Гиперспецификация означает в данном случае слишком большое количество возможных сценариев описания теории, каждое из которых будет давать полное описание предложений в ней. Возьмём в качестве языка изначальной теории любой рекурсивно перечислимый язык, и тогда описание языка в терминах его априорных компонент будет слишком полным (т.е. не будет существовать разрешающего алгоритма для определения неправильно построенных слов языка, хотя любое слово будет либо частью априорной компоненты теории, либо же нет).

	Итак, мы определились с предварительными критериями, согласно которым можно было бы уже специфицировать синтаксис некоторой теории для решения изначальной задачи. Но теперь я предлагаю принять такое исследовательское допущение, согласно которому, поздние исследования Курта Гёделя, посвящённые основаниям математики, являются продолжающими друг друга. Убедиться в этом можно открыв его работу "О до сих пор не использованном расширении финитарной точки зрения" \cite{Godel1958} 1958го года. В ней мы видим переформулировку тех же самых критериев спецификации синтаксической части теории, но уже более интересные построения. Из-за того, что перед нами не стоит цели исторической реконструкции мысли Гёделя, перейдём же сразу к этой работе.
	
	В этой работе Курт Гёдель использует понятие вычислимой функции конечного типа на натуральных числах, для задачи доказательства непротиворечивости арифметики\footnote{Также предварительно отмечается, что согласно ранее полученным результатам \cite{Godel_cont}, финитарная система может быть расширена добавлением к ней фрагментов интуиционистской логики и теории ординалов, но пока Курт Гёдель отказывается от встраивания вычислимых функций конечного типа в такую теорию, рассматривая их как альтернативу.}. Определения типов в выстроенной теории даются индуктивно, потому тип 0 это натуральные числа (\(\tau_0 := \mathbb{N}\)). Далее мы говорим, что вычислимые функции от типа \(\sigma\) к \(\tau\) — также допустимые типы \((\sigma \rightarrow \tau)\) и определим конструктор сложных типов. Если типы \(\tau_0, \tau_1, ..., \tau_k\) уже определены (где \(k \geq 1\)), то вычислимая функция типа (\(\sigma\ , \tau_0, \tau_1, ..., \tau_k)\) определяется как всегда выполнимая (и как таковая конструктивно распознаваемая) операция, которая каждому (k)-кортежу вычислимых функций типов (\(\sigma\ ,\tau_0, \tau_1, \tau_2, ..., \tau_k\)) сопоставляет вычислимую функцию типа \(\sigma\). Каждому типу соответствует множество функций \(F_\tau\), которые:
	\begin{enumerate}
		\item задаются лишь конечными конструкциями определёнными через ранее описанное правило
		\item Нормализуемы, т.е. могут быть приведены к предварённой (пренексной) нормальной форме, что обеспечивает обоснованность редукции к элементарным операциям, а также вычислимость\footnote{В более сильной интерпретации требуется выразимость в скулемовской стандартной форме, путём удаления экзистенциальных кванторов через введение скулемовских функций, но т.к. формы эквивалентно выполнимы по модели, то мы будем строго следовать Гёделю.}.
	\end{enumerate}
	Во первых, несложно заметить что перед нами предстаёт элементарная (или минимальная) теория типов, во вторых, финитистский подход исследователя сразу исключает возможность прямого включения объектов, требующих бесконечной информации, таких например как иррациональные числа. Однако, эта теория всё ещё остаётся выразительно сильной, ведь иррациональные числа допускаются в виде \emph{вычислимых приближений} — то есть как функции \(r : \mathbb{N} \to \mathbb{Q}\), удовлетворяющие условию:
	\(
	|r(n) - x| < 2^{-n}, \quad \text{где } x \in \mathbb{R}. \quad
	\)
	Такая функция \(r\) даёт рациональное приближение к иррациональному числу \(x\) с заданной точностью. Поскольку рациональные числа \(\mathbb{Q}\) могут быть закодированы через \(\mathbb{N}\) (например, при помощи пар Кантора), то такие представления остаются внутри иерархии типов конечных функций как объекты типа \(\tau_1 = \mathbb{N} \to \mathbb{N}\).
	
	Комплексные числа, в свою очередь, можно представить как пары вычислимых вещественных чисел:
	\(
	z(n) = (r_1(n), r_2(n)) \in \mathbb{Q} \times \mathbb{Q}, \quad \text{где } z = a + bi,
	\) 
	что соответствует функции \(z : \mathbb{N} \to \mathbb{Q} \times \mathbb{Q}\), или, в кодированной форме, \(z : \mathbb{N} \to \mathbb{N}\). Следовательно, вычислимые комплексные числа также допускаются как конструктивные объекты более высокого типа. 
	
	То есть теория вычислимых функций конечного типа допускает конструктивное представление алгебраических структур, таких как поля рациональных, вещественных (в виде вычислимых аппроксимаций) и комплексных чисел, через иерархию типов. При этом рациональные числа представимы напрямую, как пары целых чисел, а вещественные и комплексные числа — как процедуры (или алгоритмы), которые по запросу выдают аппроксимации с любой наперёд заданной точностью, т.е. реализуются в рамках теории как функционалы более высокого порядка, обеспечивающие конструктивный подход к классическим числовым структурам. Как мы уже уточнили ранее, это простейшая теория типов, но  в ней есть правило сведения кортежа типов к определённому типу. У Гедёля оно было единственным явно декларируемым, но для формулировки исчислений нам необходимо либо ввести правила, аналогичные исчислению секвенций, либо же ввести комбинаторную алгебру. Второй вариант является минимальным и простейшим, представим его, как он рассматривался А.Н.Трулста, в вводных заметках к работам Геделя \cite{Troelstra1995}. Нам будет достаточно определить два базовых комбинатора\footnote{Комбинатором называется просто аппликация (последовательное применение) нескольких определённых функций.}:
	\begin{align*}
		1. \quad K_{\sigma\tau} 
		&\quad \text{для каждой пары } (\sigma, \tau), \\
		&\quad \text{типа } \sigma \rightarrow \tau \rightarrow \sigma, \\[0.5em]
		2. \quad S_{\rho\sigma\tau} 
		&\quad \text{для каждого триплета } (\rho, \sigma, \tau), \\
		&\quad \text{типа } (\rho \rightarrow \sigma \rightarrow \tau) \rightarrow 
		 (\rho \rightarrow \sigma) \rightarrow \rho \rightarrow \tau.
	\end{align*}

	
	
	Но возникает проблема, ведь в соответствии с формулировкой в задачных терминах, конструктивная система необходимо должна постулировать существование таких сущностей как "решение" или "доказательство"\,, для каждой задачи относительно теории из набора \(D_S\), а также она должна предполагать расширяемость этого языка (ext), для того что бы потенциально возникающие новые задачи могли быть сформулированы на языке этой же теории. Минимальная теория типов, которую мы имеем на данный момент, не нарушает критерий moc, но нам бы хотелось что бы доказательства были частью синтаксиса. Такой переход можно сделать за два шага, во первых, в работе Х.Карри \cite{curry1958} было показано, что ранее введённые нами комбинаторы \(K_{\sigma\tau}\) и \(S_{\rho\sigma\tau}\) функционально соответствуют аксиомам импликации в системах Гильбертвоского типа(см. таблицу \ref{tab:hilbert-combinators}), В.А. Ховард же расширил это соответствие на интуиционистскую арифметику первого порядка \cite{howard1980}. Мы используем это соответствие, и вторым шагом проведём конструктивную интерпритацию логических констант в терминах доказательств, следуя Гейтингу \cite[121]{heyting1965intuitionism}.  
	\begin{table}[h]
		\small 
		\centering
		\begin{tabularx}{\linewidth}{>{\raggedright\arraybackslash}X c >{\raggedright\arraybackslash}X}
			\toprule
			\textbf{Аксиома} &
			\textbf{Комб.} &
			\textbf{Тип} \\
			\midrule
			$A \supset (B \supset A)$ &
			$K$ &
			$\sigma \to \tau \to \sigma$ \\[1ex]
			
			$(A \supset (B \supset C)) \supset ((A \supset B) \supset (A \supset C))$ &
			$S$ &
			$(\rho \to \sigma \to \tau) \to (\rho \to \sigma) \to \rho \to \tau$ \\
			\bottomrule
		\end{tabularx}
		\caption{Соответствие аксиом импликации базовым комбинаторам}
		\label{tab:hilbert-combinators}
	\end{table}
	
	Но предварительно, для реализации соответствия, расширим синтаксис введением \(\lambda\)-абстракции. Пусть \(x\) — переменная, которой приписан тип \(A\), а в расширенном контексте \(\Gamma,\,x:A\) терм \(b\) имеет тип \(B\). Тогда выражение \(\lambda x.b\) является термом типа \(A\to B\) в контексте \(\Gamma\). Семантически этот терм моделирует функцию, которая любому доказательству \(a:A\) ставит в соответствие доказательство \(b[x:=a]:B\). Таким образом все элементы типа \(A\to B\) могут быть представлены формой \(\lambda x.b\).
	Соответствие Карри–Ховарда подчеркивает, что каждое введение логической константы порождает конструктор \(\lambda\)-терма, а исключение константы соответствует вычислительному шагу (редукции) над этими термами. Так, введение импликации есть построение лямбда-абстракции \(\lambda x.M\), где \(M\) — доказательство \(B\) из \(A\); а исключение — это применение \(\beta\)-редукции: из \((\lambda x.M)\,N\) получается \(M[x:=N]\) \cite[46]{ershov_celishev2012}. Следовательно, проверка корректности вывода эквивалентна проверке корректной типизации \(\lambda\)-терма: если \(M\) имеет тип \(A\to B\) и \(N\) имеет тип \(A\), то применение \(MN\) имеет тип \(B\). Более формально это записывается как \(M : A \to B\) и \(N : A\) следует, что \(MN : B\). 
	
	Конъюнкция \(A\wedge B\) интерпретируется как тип произведения \(A\times B\) с конструктором \(\langle M,N\rangle\), дизъюнкция \(A\vee B\) — как тип суммы \(A + B\) с конструкторами левой и правой инъекции, а \(\bot\) — как пустой тип \(\varnothing\) с оператором \(\varnothing\to C\) для любого \(C\). В общем виде соответствие Карри–Ховарда сводит высказывание к типу и доказательство к терму этого типа. Любой валидный вывод \(A\vdash B\) влечёт существование \(\lambda\)-терма \(M\) типа \(A\to B\), а сам вывод — это \(\beta\)-редукция термов.	
	
	Уточним, что такое понимание семантической части теории отличается от традиционного по Тарскому, называемого теоретико-модельным. Модельная семантика определяет смысл формулы "извне" при помощи структур (моделей). Каждая формула \(A\) сопоставляется семантическому условию \(\mathcal{M}\vdash A\), означающему, что \(A\) истинно в модели \(\mathcal{M}\). Логическое следование \(\Gamma\models B\) тогда означает, что во всех моделях \(\mathcal{M}\), где истинны все формулы \(\Gamma\), истинна и \(B\). Эта внешняя точка зрения опирается на "истинность во всех моделях" и требует проверки существования моделей (или, напротив, контрпримеров) \cite[27]{Schroeder-Heister2024}. При этом семантические конструкции (домен, функция интерпретации предикатов) неявно предполагают принцип исключённого третьего: либо \(\mathcal{M}\models A\), либо \(\mathcal{M}\not\models A\). Конструктивность при этом не гарантируется: формула может быть "полупроверяемой" — не имея доказательства, но будучи истинной в некоторых моделях.	Теоретико-доказательная же семантика, которой мы и будем пользоваться, задаёт смысл "изнутри"\  , через синтаксисические правила. В первом случае "валидность" определяется универсальностью истинности (семантическим требованием), во втором — "валидность" означает существование конструктивного метода (правил, \(\lambda\)-терма), преобразующего доказательство предпосылки в доказательство заключения. Такая синтаксическая картина тождественна "наличию доказательства" (терма данного типа). Теоретико-доказательная семантика позволяет строить корректно типизированные \(\lambda\)-термы (программы) вместо поиска внешних моделей, обеспечивая полную конструктивность и явное соответствие между доказательством и вычислением. 
	
	Теперь мы можем проследить конструктивное объяснение логических констант (дизъюнкции, импликации, отрицания) в доказательных терминах данное Гейтингом \cite[123]{heyting1965intuitionism}. Так, доказательство импликации  
	\(\,A\supset B\,\)  
	трактуется как функция, преобразующая любое доказательство \(A\) в доказательство \(B\), что формально фиксируется отождествлением  
	\(\,A\supset B\longleftrightarrow A\to B\,.\)  
	Доказательство конъюнкции  
	\(\,A\wedge B\,\)  
	стандартно представляется, что мы уже уточнили ранее, в виде упорядоченной пары \((a,b)\), где \(a:A\) и \(b:B\), что даёт  
	\(\,A\wedge B\longleftrightarrow A\times B\)\footnote{Произведение \(A\times B\) есть тип пар \((a,b)\), где \(a\in A\) и \(b\in B\).}. Дизъюнкция  
	\(\,A\vee B\,\)  
	конструктивно истинна тогда и только тогда, когда имеется доказательство одного из дизъюнктов с указанием, какой именно; это фиксируется отождествлением  
	\(\,A\vee B\longleftrightarrow A + B\)\footnote{Тип \(A + B\) формально определяется как дизъюнктное объединение двух множеств с метками:
		\[
		A + B \;=\; \{(0,a)\mid a\in A\}\;\cup\;\{(1,b)\mid b\in B\}.
		\]
		Операции инъекции задаются так:
		\(
		\quad \mathrm{inl}(a) \;=\; (0,a),\quad
		\mathrm{inr}(b) \;=\; (1,b). \quad
		\)
		Здесь "левая метка" \(0\) и "правая метка" \(1\) различают элементы, пришедшие из \(A\) и \(B\), что позволяет однозначно восстановить, из какого слагаемого произошло данное значение.}. Отрицание определяется как  
	\(\,\neg A \equiv A \supset \bot\,\),  
	где \(\bot\) есть пропозиция "абсурдности" без доказательств, что даёт  
	\(\,\neg A \longleftrightarrow A \to \varnothing\,.\)
	
	Но для интерпретации кванторов Гейтинг вводит бесконечно продолжающиеся последовательности, обобщённые в понятие финитарного потока, используемого для формулировки и доказательства теоремы о веерах \cite[128]{heyting1965intuitionism}. Сама интерпретация получается сильной, но из-за громоздкости пути к ней, мы поступим иначе, и введём понятие семейства множеств. Пусть \(A\) — некоторое фиксированное множество. Тогда \emph{семейство множеств над \(A\)} — это отображение
	\(
	B : A \longrightarrow \mathsf{Type},
	\)
	то есть такая функция, которая каждому элементу \(a\in A\) сопоставляет некоторое множество \(B(a)\). Говорят, что \(B\) "зависит" от \(a\), и такие конструкции называются \emph{зависимыми типами}.
	Интуитивно можно думать о \(B\) как о переменной, которая для каждого значения \(a\) принимает своё собственное множество возможных значений. Например, пусть
	\(
	A = n, n \in \mathbb{N}, \quad
	B(n) = \{0, 1, \dots, n\},
	\)
	тогда \(B\) — это семейство конечных отрезков, зависящих от натурального числа \(n\). Для \(n = 3\) получим \(B(3) = \{0,1,2,3\}\), а для \(n = 0\) — \(B(0) = \{0\}\).	 Соответственно, экзистенциальный квантор  \((\exists x \in A)\,B(x)\) теперь может быть отождествлён с типом \(\Sigma_{x:A}B(x)\), т.е. множеством всех пар \((a, b)\), где \(a\in A\), \(b\in B(a)\). Соответственно квантор всеобщности отождествляется с семейством функций 
	\(
	\quad (\forall x\in A)\,B(x)\;\longleftrightarrow\;\Pi_{x\!:\!A}B(x),
	\)
	который определяется как множество всех функций \(f\), каждая из которых каждому элементу \(a\in A\) ставит в соответствие элемент \(f(a)\in B(a)\).
	\newline Формально:
	\(
	\Pi_{x:A}B(x)=\{\,f \mid \forall a\in A,\;f(a)\in B(a)\,\}.
	\)
	
	Такая семантика пропозиций как множеств доказательств непосредственно переходит в синтаксис интуиционистской теории типов М. Лёфа \cite{martinlof1984intuitionistic} (MLTT)\footnote{MLTT это используемая далее для краткости аббревиатура от Martin-Löf Type Theory.}, которую мы и будем использовать как сильную теорию, в рамках решения первоначальной задачи.	

	\subsection*{2.2. Описание сильной теории}
	\addcontentsline{toc}{subsection}{2.2. Описание сильной теории}
	Интуиционистская теория типов Мартина-Лёфа обогащена конструкциями, достаточными для формализации отношения выводимости и предиката доказуемости слабых подсистем внутри себя, и вместе с тем содержит {\it слабую подсистему} \(S\subset\mathrm{MLTT}\), в которой исключены некоторые мощные средства (W-типы), но сохраняются базовые типовые операции. Ниже опишем минимальную систему интуиционистской теории типов, которую назовём ядром.
	
	В ядре MLTT используются следующие формы суждений: 	\(\Gamma, \) \newline
	\(
	\Gamma\vdash A\;(\mathrm{type}),
	\quad
	\Gamma\vdash t : A; \quad
	\) 	где контекст \(\Gamma\) представляет собой упорядоченный список допущений вида 
		\(
		x_1\!:\!A_1,\;x_2\!:\!A_2,\;\dots,\;x_n\!:\!A_n,
		\) с требованием, что при добавлении каждого \(x_i:A_i\) уже выполнены все предыдущие контексты \(\Gamma_{i-1}\vdash A_i\;\mathrm{type}\). 
		\(\Gamma\vdash A\;\mathrm{type}\) означает, что в контексте \(\Gamma\) выражение \(A\) является корректным типом.  
		\(\Gamma\vdash t:A\) означает, что \(t\) является корректным термом (доказательством) типа \(A\) в контексте \(\Gamma\). Каждый конструктивный тип задаётся тремя (или четырьмя) правилами, а именно:
	\begin{enumerate}[leftmargin=1.5em]
		\item \emph{Правило формирования}, определяющее, каким образом строится новый тип.
		\item \emph{Правило введения}, задающее, как строятся термы этого типа.
		\item \emph{Правило устранения}, определяющее способ "использовать" или "распаковать" термы этого типа.
		\item (Опционально) \emph{Правило вычисления}, показывающее, как введённые термы "сворачиваются" к более простому виду (редукция).
	\end{enumerate}
	В данной работе не приводится всех правил, они могут быть найдены в  \cite[13-36]{martinlof1984intuitionistic}, но заметим одну важную вещь о вырождении типов, помогающую лучше понять построение доказательств.
	\(\Pi\)\nobreakdash-тип квантификации по всем элементам и функциональный тип следования ("\(\to\)") связаны следующим образом: 
	\(A\to B \cong \Pi_{x:A}B \quad (\text{когда }B\text{ не зависит от }x) \), т.е. интерпретация квантора всеобщности вырождается в обычное следование, при условии что область значений функции не зависит от аргументов функции. В качестве примера можно привести функцию, которая принимает в качестве аргумента натуральное число, удваивает его, и возвращает 1, если оно чётно и 0 в противном случае. Очевидно, что эта функция всюду вычислима и является константной.
	Аналогично, \(\Sigma\)-тип зависимой суммы вырождается в прямое произведение:
	\(
	A\times B \cong \Sigma_{x:A}B \quad (\text{когда }B\text{ не зависит от }x)
	\), и в качестве иллюстрации, мы теперь можем взять ту же самую функцию, которая принимает любое натуральное число в качестве аргумента и возвращает 1, если оно чётно и 0 в противном случае. Также очевидно, что всем натуральным числам теперь будут сопоставлены пары чисел, в которых второе будет свидетельствовать о чётности первого.
	
	В MLTT 1984 года также вводится иерархия универсумов \(\mathcal{U}_0,\mathcal{U}_1,\dots\), где каждый \(\mathcal{U}_n\) является типом, элементы которого — это типы из \(\mathcal{U}_{n-1}\) \cite[13]{martinlof1984intuitionistic}. В первоначальной версии MLTT 1971 г. существовало правило образования типа \(\Pi_{\rho:*}\rho\), где \(*\) был универсальным типом, что эквивалентно \(\forall \rho:*,\rho\) и ведёт к парадоксу, согласно которому можно было построить утверждение, что каждый тип населён\footnote{Т.е. имеются объекты этого типа.}, и как следствие, каждое суждение доказуемо. Парадокс представил Жан-Ив Жирар \cite{girard1972paradox}, после чего в редакции MLTT 1984 г. была введена иерархия универсов для его устранения. Универсы позволяют говорить о "типах как объектах" и строить кодировки целых семейств типов внутри теории.
	 
	Так как в теории у нас появляются зависимые типы, которые могут зависеть как от предыдущих типов, так и от термов, нам необходимо иметь возможность проверять, равны ли два терма по определению (т.е. совпадает ли их тип). Но так как мы можем составить два эквивалентных типа разными путями (но всё ещё через последовательное применение правил), эта задача становиться не такой тривиальной. Её решение -- это иметь некоторое правило сворачивания типов, благодаря которому мы могли бы привести их к одной форме и соответственно убедиться в их эквивалентности. Таким правилом является \(\beta\)-редукция. Работу его можно объяснить таким образом, что в ядре MLTT фиксируются все возможные \emph{редексы} (сокращаемые подвыражения) и устанавливается способ их "свёртки" к более простым термам. Под редексом понимается любой участок терма, к которому применимо правило свёртки: в функциональных типах это выражения вида \((\lambda x.\,t)\,a\) или в произведениях \(\Pi_i(\langle a,b\rangle)\). Именно эти редексы последовательно заменяются на результаты свёртки:
	\(
	 (\lambda x.\,t)\,a \;\longrightarrow\; t[x:=a],
	 \quad
	 \Pi_1\bigl(\langle a,b\rangle\bigr)\;\longrightarrow\;a,
	 \quad
	 \Pi_2\bigl(\langle a,b\rangle\bigr)\;\longrightarrow\;b
	\). Соответственно процесс последовательного применения \(\beta\)-редукций к любому терму \(t\) до тех пор, пока не останется ни одного редекса, то есть пока не будет достигнута \emph{нормальная форма} \(t_{\mathrm{nf}}\) называется \textit{нормализацией}. Терм считается нормализованным, если ни к одному его подвыражению нельзя более применить ни одного правила свёртки. В ядре MLTT доказано, что любая последовательность свёрток оканчивается за конечное число шагов (свойство \emph{сильной нормализации}), и что конечный результат не зависит от того, в каком порядке выбирались редексы. Формально, \emph{Сильная нормализация} означает, что не существует бесконечной цепочки
	 	\(
	 	t \;\longrightarrow\; t' \;\longrightarrow\; t'' \;\longrightarrow\; \cdots. \quad
	 	\)
	 	Другими словами, любое начальное выражение \(t\) "схлопывается" к некоторому нормальному виду \(t_{\mathrm{nf}} \quad\) за конечное число \(\beta\)-редукций. Также в ядре MLTT выполняется теорема Чёрча-Россера (что означает конфлюэнтность ядра MLTT), которая гарантирует, что если один и тот же терм \(t\) можно свёрнуть двумя (или более) разными последовательностями \(\beta\)-редукций к двум формам \(t_1\) и \(t_2\), то существует общий терм \(t_3\), к которому свёртываются как \(t_1\), так и \(t_2\). Это обеспечивает единственность нормальной формы: независимо от выбора редексов мы придём к одному и тому же \(t_{\mathrm{nf}}\).  
	 	 
	 Теперь вспомним об обязательном условии, выдвигаемом Гёделем для слабой подсистемы -- она должна быть непротиворечива, иначе всё это не имело бы смысла. Мартин Лёф говорит о непротиворечивости интуиционистской теории типов в двух смыслах, а именно метаматематическом и "простодушном" \cite[38]{martinlof1984intuitionistic}. От семантического обоснования аксиом и правил вывода в Гильбертовском смысле автор отмахивается, вместо этого предлагая простое рассуждение. Поскольку в ядре MLTT тип \(\bot\) не имеет правил введения (нет конструкторов), не существует ни одного терма \(t\) с суждением  \(\vdash t:\bot\). Если бы такое доказательство существовало, то по свойству сильной нормализации его нормальная форма \(t_{\mathrm{nf}}\) тоже имела бы тип \(\bot\), однако нормальная форма не может содержать конструкторов для \(\bot\). Следовательно, \(\vdash t:\bot\) неприменимо, и ядро MLTT не выводит противоречиво пустое высказывание. Таким образом, непротиворечивость ядра MLTT непосредственно вытекает из свойств нормализации и конфлюэнтности системы \footnote{В MLTT доказательство сильной нормализации и теоремы Чёрча-Россера может быть проведено аналогично данному в работе \cite{lambek1986introduction}, а обобщённая конструктивная модель для подтверждения непротиворечивости MLTT может быть найдена в \cite{nordstrom1990semantics}.}. 
	 
	\subsection*{2.3. Ограничения конструктивного подхода и неожиданные результаты}
	\addcontentsline{toc}{subsection}{2.3. Ограничения конструктивного подхода и неожиданные результаты}
	Несмотря на выразительность MLTT как формальной системы, её конструктивная природа накладывает существенные ограничения. MLTT базируется на интуиционистской логике, соответственно многие тактики\footnote{Термин используемый в системах автоматической проверки доказательств, в частности в coq, означающий множество конструктивных команд (управляющих конструкций языка) над термами, типами и контекстом, позволяющие строить доказательства в такой форме, что бы оно могло быть проверено на корректность. При этом тактики реализуются на метауровневом языке, помогая строить итоговый терм-доказательство определённого типа, который уже и проверяется ядром coq.} проведения доказательств не соответствуют традиционным математическим, к примеру доказательство путём разбора по случаям допускается только с указанием того, какой из дизъюнктов истинен, т.е. лишь конструктивно, если же дизъюнкция принципиально недоказуема конструктивно и появляется желание применить закон исключённого третьего, то мы попадаем в ловушку, в связи с интуиционистской природой системы. Конечно и этого можно избежать, явно введя закон исключённого третьего, но тогда мы теряем конструктивность проверки доказательства и соответственно утверждение правильности доказательства становится не фактом, а скорее событием, имеющим вероятностную природу, терм может быть проверен ядром, но не может быть выполнен или интерпретирован как алгоритм.
	
	Теперь уже нетрудно заметить, что каждый шаг принятый (или описанный) нами в построении формальной онтологии, служивший целью приблизиться к воплощению финитистской программы Гильберта, в итоге дал совсем не те результаты, которые мы ожидали. Так оказалось, что MLTT, полностью удовлетворяющая всем критериям Гёделя, не предназначена для построения системы оснований математики. В её основе лежит не попытка внешнего обоснования системы, а внутренняя интерпретация правил построения, где корректность исходит из локальной согласованности конструктивных процедур. Сама система не предлагает способа выразить или доказать собственную непротиворечивость; вместо этого она полагается на семантическую интерпретацию — правила вывода принимаются как самодостаточные, аналогично тому как значение знака в языковых играх Витгенштейна определяется его применением. В этом смысле MLTT ближе к грамматике, чем к метатеории: она не утверждает истинность всех теорем, но задаёт язык, на котором возможно формулировать доказательства и проверять знания\footnote{Но не всё так печально, ведь интуиционистская теория типов послужила минимальной моделью для программы унивалентных оснований математики, предложенной В.  А. Воеводским. Возможность автоматической проверки доказательства оказалась очень плодотворной в теоретическом плане, но новые основания математики базируются на гомотопическом расширении изначальной теории типов \cite[90]{HoTTBook2013}, описание которого требует довольно долгого погружения в математические аспекты этой теории, поэтому ограничимся лишь упоминанием этой успешной программы.}.
	
	Также попытки интерпретации MLTT в терминах внешней непротиворечивости всегда относительны: используют модели в CZF\footnote{Constructive Zermelo-Fraenkel, аксиоматическая теория множеств, сформулированная на базе интуиционистской логики, с требованием конструктивности всех утверждений существования.} \cite{aczel1978typetheoretic}, иерархии универсумов интерпретируются через большие кардиналы или другие сильные теории. Иными словами, непротиворечивость MLTT всегда выводится через мост к другой системе, никогда не замыкаясь на себя. При этом нельзя забывать и о теореме Гёделя: любая достаточно выразительная теория, способная формализовать арифметику, не может доказать собственную непротиворечивость. В MLTT это ограничение особенно заметно: её конструктивный характер не допускает классических приёмов, вроде доказательства от противного, и тем самым делает невозможной саморефлексивную метатеорию. Если сравнить MLTT с навигационной системой, то она не указывает точку старта — только маршрут. Её задача не в том, чтобы показать, что весь математический ландшафт непротиворечив, а в том, чтобы гарантировать безопасность движения по заранее заданным координатам. Поэтому она выступает как инструмент точной локальной верификации, а не как высшая инстанция, гарантирующая всеобщую непротиворечивость.
	
	\subsection*{2.4. Расширенные модели теории}
	\addcontentsline{toc}{subsection}{2.4. Расширенные модели теории}
	Наличие нормальных форм напрямую связано с алгоритмической проверяемостью системы, о которой мы уже упомянули ранее. Свойство нормализации приводит к разрешимости всех выводимых суждений: для любой корректной тройки \((\Gamma,T,t)\) существует алгоритм, решающий, выводимо ли \(\Gamma\vdash t:T\). Иными словами, проверка того, что терм \(t\) имеет тип \(T\) в контексте \(\Gamma\), может быть автоматизирована (эта процедура называется type checking). На этой основе создаются пруверы, или, говоря более формально, ассистенты автоматической проверки доказательств (proof assistants), реализующие MLTT. Например, в работе \cite{adjedj2023martinlofalacoq} приводится механизация метатеории MLTT в прувере Coq: она демонстрирует разрешимость конверсии и проверку типов и позволяет получить конструктивный выполняемый чекер\footnote{Программу автоматической проверки доказательства.} MLTT с поддержкой \(\Pi\)-, \(\Sigma\)-типов, натуральных чисел и, не рассмотренных нами, классов идентичностей. Аналогично, система Agda основана на интуиционистской MLTT и имеет алгоритмически разрешимые правила вывода. В итоге строгость синтаксиса и конструктивная семантика MLTT, сочетающиеся с нормализацией, гарантируют, что любое математическое доказательство, заданное как MLTT-терм, может быть проверено с помощью машины: корректность вывода эквивалентна проверке корректности терма по правилам типизации. Знание в таком подходе приобретает машинно проверяемую структуру, где каждое утверждение имеет чёткий синтаксический и семантический статус. Это делает MLTT не просто логико-математическим инструментом, а полноценной вычислимой онтологией, в которой вся совокупность знаний может быть подвергнута формальной валидации. Но проверка доказательств это довольно узко-специализированное свойство MLTT, интересное в большей степени практикующим математикам, существуют также и другие способы использовать интуиционисткую теорию типов для репрезентации знания. 
	
	Уже было упомянуто, что MLTT имеет множество моделей, так вот она также имеет категорную интерпретацию.
	Категориальный подход к интерпретации системы типов порождает формальную онтологию репрезентации знаний, что позволяет устранить традиционный разрыв между знанием и его обоснованием и одновременно обеспечивает автоматизацию проверки доказательств. Действительно, для того чтобы представить не только научные, но и обыденные сведения, необходима процедура обоснования, аналогичная "знанию как истинному и обоснованному мнению" (см. статью Э. Геттиера \cite{gettier1963}). Стандартные архитектуры KR не поддерживают подобные обосновывающие процедуры: хотя с помощью теории первого порядка можно формализовать само знание (то есть построить онтологию, описывающую факты и отношения между ними), процедуры обоснования в таком случае требуют ресурсов логики второго порядка. Тем не менее, когда речь идёт о формальной проверке обоснованности умозаключений (то есть проверке доказательств), оказывается достаточным аппарата теории первого порядка, поскольку формализация математических доказательств и их оправданий при помощи вычислимых исчислений давно вошла в практику. При этом возникает вопрос: в какой мере результаты машинной проверки могут считаться полноценным обоснованием, достаточным для убеждения и, следовательно, для приобретения знания? Корень этой проблемы заключается в том, что человек не всегда способен проследить все шаги машинного доказательства из-за их объёма и неочевидной структуры, а также из-за отсутствия "стройности" в изложении (пункт, на котором особенно акцентирует внимание Грегори Чайтин в своей пятой главе \cite{Chaitin1999TheU}).
	
	Проблема недоступности машинных свидетельств и критерия стройности приводит к необходимости построения модели представления знаний, которая одновременно позволяла бы организовать автоматическое доказательство и сохраняла бы доступность для человеческого анализа. Именно такой модели отвечает категорная семантика теории типов: она оперирует только средствами первого порядка, но благодаря своей внутренней структуре позволяет формализовать и машинную проверку, и "удобочитаемую" форму доказательства. Словом, построение формальной онтологии на базе категориальных конструкций даёт возможность одновременно обеспечить (1) существование вычислимого механизма проверки доказательств и (2) интуитивно понятную схему для человеческого восприятия доказательств, не прибегая к средствам логики второго порядка.
	
	Практическая разработка подобной онтологии ведётся уже несколько лет. В частности, система Coq, традиционно использующая конструкцию теории типов Мартина Лёфа для автоматизации построения и верификации доказательств, в настоящее время перерабатывается на основе категориальных представлений. Авторы этих проектов реализуют Coq "с нуля"\  , адаптируя его ядро к категориальной модели, что позволило бы объединить проверяемость машинным способом и сохранение стройного, понятного человеку изложения. 
		
	При построении категорной модели MLTT основная сложность состоит в представлении операторов зависимых типов \(\Pi \text{ и } \Sigma\), что может быть проведено благодаря введению условия Бека-Чевалли \cite[§14]{Pavlovic1991}. Используя малые категории и язык теории категорий, можно формализовать MLTT как систему репрезентации знания, в которой контексты интуитивно понимаются в качестве состояния знания: совокупность тех предположений и фактов, которые на данный момент "установлены". Когда мы переходим от одного контекста к другому, мы либо добавляем новые гипотезы (расширяем знание), либо отбрасываем некоторые (сужаем круг допущений).	
	В таком подходе "типы в контексте" рассматриваются как семейства понятий, зависящих от текущего набора допущений. Термы этих типов — это конкретные объекты или доказательства, соответствующие тем или иным утверждениям, причём всё это меняется в зависимости от того, какие базовые факты мы уже принимаем.
	
	Два ключевых оператора зависимых типов, $\Sigma$ и $\Pi$, задают способы концептуализации (существует класс элементов) и абстрагирования (для всех элементов некоторых классов) знания соответственно. Важно, что при переносе знаний из одного контекста в другой (скажем, когда мы поменяли часть допущений) сборка "существует" и "для всех" работает согласованно — то есть неважно, в каком порядке мы расширяем контекст и применяем $\Sigma$ или $\Pi$, результат будет тем же (это в целом и есть условие Бека–Чевалли, которое согласно работе \cite{Lamarche2014} совершенно незаметно в синтаксисе теории типов, но эксплицируется и играет важную роль в категориальном контексте).
	При этом таксономия классов, порождённая интернализацией (т.е. пропозиции и их доказательства являются синтаксической частью теории) позволяет представлять любой сложный объект как один формализм, разрешимость которого выяснима за конечное количество шагов. Также, вместо долгого и сложного построения выводов мы теперь имеем возможность строить диаграммы между контекстами, и рассматривать условия при которых они являются коммутативными. Также такой визуальный подход помогает нам в построении сложных семантических сетей и топологии кластеров баз данных, что является довольно нетривиальной задачей в области прикладной репрезентации знаний, ведь обычно, реляционные базы данных проектируют на нескольких уровнях (объектном, логическом и физическом) и задачу соответствия представлений объектов как раз легко решать с помощью категориой модели MLTT.
	
	\newpage
		
	\section*{Заключение}
	\addcontentsline{toc}{section}{Заключение}

	Системный подход к рассмотрению ключевой проблемы в философии математики XX века помог нам сформировать некоторый описательный язык, использование которого по отношению к математическим теориям позволяет распространить их применение на многие сферы вне чистой науки. В ходе исследования была применена комбинация историко‐философского анализа, формализационного метода и методологии математического структурализма, что позволило продемонстрировать эволюцию взглядов на природу математического знания и обосновать разработку формальной онтологии как единой опосредующей "грамматики" для построения математических объектов и доказательств. 
	
	Сначала была проведена детальная историческая реконструкция "третьего кризиса" оснований, включающая анализ публикаций начала‐середины XX века, в частности работ Д. Гильберта, К. Гёделя, А. Тарского и других. Это позволило показать, что парадоксы теории множеств и теоремы неполноты выдвинули проблему непротиворечивости математических систем на новый уровень: стало ясно, что классические попытки финитарного обоснования неизбежно упираются в ограничения внутренних ресурсов формальных языков. Выявление этой мета‐проблемы требовало не только технического исследования аксиоматических систем, но и философского осмысления онтологических предпосылок, лежащих в основе математических конструкций. В частности, был осуществлён методологический анализ ключевых текстов XX века с точки зрения того, какие онтологические обязательства они подразумевают и каким образом они на практике выдвигали необходимости трансформации парадигмы восприятия математических объектов.
	
	Философская сторона работы опиралась на идеи математического структурализма, выраженные в трудах Р. Дедекинда, Э. Нётер, С. Маклейна и других, а также на интуиционистский взгляд Л. Брауэра и М. Лёфа. Структуралистская перспектива позволила отказаться от сущностного представления чисел и множеств и сфокусироваться на том, что математическое содержание задаётся отношениями и операциями внутри некоторой абстрактной структуры. С этой точки зрения ключевым становится принцип "изоморфные объекты—равны"\  , все, что важно, — это не "сущность" объекта, а его позиция в структуре. 
	
	После, вместе с историческим описанием идей математического структурализма, одновременно применялся формализационный подход, в рамках которого математические объекты рассматривались как синтаксические конструкции: типы, термы и правила преобразования. Общая идея заключалась в том, чтобы перейти от классического понимания модели как внешней семантической структуры, удовлетворяющей аксиомам—к "внутреннему" пониманию модели как совокупности синтаксических доказательств, репрезентированных \(\lambda\)\nobreakdash‐термами. Для этого нами была разработана схема "примитивы + грамматика"\  , где "примитивы" представляют собой минимальный набор символов (типовых конструкторов, переменных, констант), а "грамматика" описывает правила их комбинирования, применения индуктивных схем, операций редукции и типизации. Важнейшим отличием такого подхода является то, что он не требует внешнего оценивающего уровня, способного сравнить одну модель с другой: здесь каждая интерпретация превращается в корректный \(\lambda\)‐терм, а проверка корректности сводится к синтаксической нормализации и проверке типов. Таким образом, была разработана формальная онтология, отвечающая одновременно трём важным критериям: конструктивности (все правила редукции и типизации алгоритмически проверяемы), минимальных онтологических обязательств (moc) и расширяемости (ext). 
	
	При описании формальной онтологии основное внимание уделялось тому, как синтаксические примитивы могут быть сведены к правилам, гарантирующим нормализацию и конечность. Описанные в работе "минимальные онтологические обязательства" означают, что в лексику включаются лишь те элементарные понятия, которые можно эффективно реализовать и проверить в рамках одного языка. При этом критерий расширяемости позволял демонстрировать, что можно добавлять новые типовые конструкторы, например, для задания естественных чисел индуктивным типом, без нарушения общей согласованности системы. Важная техническая деталь заключалась в том, что при введении новых конструкций соблюдается правило "никаких скрытых невычислимых объектов": все определения должны оставаться конструктивными, а финитная проверка каждого шага доказательна за конечное число операций.
	
	Конструктивная интерпретация была проиллюстрирована на примере интуиционистской теории типов Мартина‐Лёфа (MLTT). Методика её исследования включала в себя синтаксический анализ правил формирования типов $\Pi$ и $\Sigma$, проверку свойств нормализации и теорем о непротиворечивости базовых подсистем. Были рассмотрены способы представления натуральных чисел как индуктивного типа, рекурсивного задания вещественных чисел через последовательности рациональных приближений и конструктивного построения логических констант. Для формальной проверки корректности вводимых правил использовался метод "доказательство—\(\lambda\)‐терм—тип" (соответствие Карри–Ховарда), в котором каждому доказательству сопоставляется \(\lambda\)‐терм, а тип этого терма соответствует утверждению. В результате аналитическая часть доказала, что базовая версия MLTT удовлетворяет условиям конечности и нормализации, что делает возможным внутри неё проверять доказательства большей части математических утверждений.
	
	Наконец, следует отметить, что разработанные здесь методы и подходы имеют потенциал применения за пределами "чистой" математики. Формальная онтология, основанная на MLTT, может быть использована для построения онтологических баз знаний в компьютерных системах, где формализованные математические структуры служат основой для верификации требований и корректности программного обеспечения. В частности, идея представления доказательств в виде термов позволяет интегрировать математические обоснования и логические спецификации в рамках единого синтаксического языка, что упрощает процессы автоматизированной проверки. Категорная же модель интуиционистской теории типов может применяться как дескриптивная онтология и служить инструментом описания знания в динамическим системах.
	
	\newpage
	
	\printbibliography[heading=bibintoc,title={Список литературы}]
	%\addcontentsline{toc}{section}{Список литературы}
	
\end{document}