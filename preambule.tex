\usepackage[T2A]{fontenc}
\usepackage[utf8]{inputenc}
\usepackage[russian]{babel}

\usepackage{amsmath, amssymb, amsfonts}
\usepackage{geometry}
\usepackage{csquotes}
\usepackage{hyperref}
\usepackage{setspace}
\usepackage{bussproofs}
\usepackage{enumitem}
\usepackage{xcolor} % для выделения цветом

\usepackage{graphicx}
\usepackage{float}            % для [H]
\usepackage{tikz}     
\usepackage{tikz-cd}        % для рисунков
\usepackage{tabularx}
\usepackage{array} % для \arraybackslash
\usetikzlibrary{arrows.meta, positioning, shapes.geometric}
\usepackage{stmaryrd} 
\providecommand\nobreakdash{\hbox{-}\nobreak\hskip0pt}
\allowdisplaybreaks

\usepackage{caption}          % красиво оформляет подписи
\usepackage{booktabs}         % улучшенная верстка таблиц
\sloppy

\renewcommand{\footnotesize}{\fontsize{12pt}{11pt}\selectfont}

\usepackage[
backend=biber,
style=gost-numeric,		%ext-numeric, можно также 'gost-authoryear'
language=russian,
sorting=langnty
]{biblatex}

\addbibresource{bib/kursach.bib} % путь к твоему .bib файлу

% Разрешаем проверять язык записи (langid)
\usepackage{etoolbox}

% Для одиночного постнота (например, \cite[12]{key}):
\DeclareFieldFormat{postnote}{%
	\iffieldequalstr{langid}{russian}
	{c.~#1}% если langid = russian → «с. <номер>»
	{p.~#1}% иначе → «p. <номер>»
}

% Для множественного постнота (например, \cite[12, 34]{key}):
\DeclareFieldFormat{multipostnote}{%
	\iffieldequalstr{langid}{russian}
	{c.~#1}% «с. 12, 34»
	{p.~#1}% «p. 12, 34»
}


% ГОСТ сокращения городов
\DeclareListFormat{location}{%
	\ifstrequal{#1}{Москва}{М.}%
	{\ifstrequal{#1}{Санкт-Петербург}{СПб.}%
		{\ifstrequal{#1}{Ленинград}{Л.}%
			{\ifstrequal{#1}{Минск}{Мн.}%
				{\ifstrequal{#1}{Новосибирск}{Новосиб.}%
					{#1}}}}}%
}

\newcommand{\mycomma}{\addcomma\space}

% --- BOOK ---
\DeclareBibliographyDriver{book}{%
	\printtext{\mkbibemph{\printnames{author}}}%
	\mycomma
	\printfield{title}%
	\mycomma
	\printlist[location]{location}
	\mycomma
	\printfield{year}%
	\finentry
}

% Аналогично для остальных драйверов:

\DeclareBibliographyDriver{article}{%
	\printtext{\mkbibemph{\printnames{author}}}%
	\mycomma
	\printfield{title}%
	\mycomma
	\printfield{journaltitle}%
	\mycomma
	\printlist[location]{location}
	\mycomma
	\printfield{year}%
	\finentry
}


% --- INCOLLECTION ---
\DeclareBibliographyDriver{incollection}{%
	\printtext{\mkbibemph{\printnames{author}}}%
	\mycomma
	\printfield{title}%
	\mycomma
	\printfield{booktitle}%
	\mycomma
	\printlist[location]{location}
	\mycomma
	\printfield{year}%
	\finentry
}

% --- INBOOK ---
\DeclareBibliographyDriver{inbook}{%
	\printtext{\mkbibemph{\printnames{author}}}%
	\mycomma
	\printfield{title}%
	\mycomma
	\printfield{booktitle}%
	\mycomma
	\printlist[location]{location}
	\mycomma
	\printfield{year}%
	\finentry
}

% --- MISC ---
\DeclareBibliographyDriver{misc}{%
	\printtext{\mkbibemph{\printnames{author}}}%
	\mycomma
	\printfield{title}%
	\mycomma
	\printlist[location]{location}
	\mycomma
	\printfield{year}%
	\finentry
}

% --- INPROCEEDINGS ---
\DeclareBibliographyDriver{inproceedings}{%
	\printtext{\mkbibemph{\printnames{author}}}%
	\mycomma
	\printfield{title}%
	\mycomma
	\printfield{booktitle}%
	\mycomma
	\printlist[location]{location}
	\mycomma
	\printfield{year}%
	\finentry
}

\DeclareSortingTemplate{langnty}{
	% 2.1. Сортируем по langid (строку "russian" < "english")
	\sort[direction=descending]{
		\field{langid}
	}
	% 2.2. Внутри каждой языковой группы — по автору (editor, translator, sortname)
	\sort{
		\field{author}
		\field{editor}
		\field{translator}
		\field{sortname}
	}
	% 2.3. Затем — по году (date)
	\sort{
		\field{year}
		\field{date}
	}
	% 2.4. И, наконец, по названию (title)
	\sort{
		\field{title}
	}
}


\geometry{
	a4paper,
	top=2cm,
	bottom=2cm,
	left=3cm,
	right=1.5cm
}

\setlength{\parindent}{12.5mm}
\setlength{\parskip}{0.7em}
\onehalfspacing
\newcommand{\Con}{\mathrm{Con}}
